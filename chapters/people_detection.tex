\chapter{People detection}
\label{chp:people_detection}
% Explain:
% - Why this is needed
% - How it is implemented
% - Rx beamforming
% - How it works, maybe some points on the accuracy and how it performs on different types of persons

The first step in dynamic multiple people vital signs estimation, is detecting and tracking persons in range of the sensor. In this chapter, the algorithm to detect persons is explained.

\section{Implementation}
There are existing algorithms to group together multiple points with high energy which are close to each other. These groups of points are an object, a person or any other thing in the space which reflects the RF waves from the sensor. One example is a Constant False Alarm Rate (CFAR) algorithm. This algorithm is already implemented in the mmWave SDK~\cite{mmwavesdk_website}. This is a very difficult algorithm to set up however, it requires intricate knowledge about the inner workings of the algorithm and of all of the right data streams in the chip. Documentation on how to implement this algorithm was missing, only a working general version is provided. Because of these reasons, a detection algorithm was build from scratch. Because it was build from the ground up, it was ensured that all of the needs and output formats could be met.

\subsection{Input data format}
The algorithm returns a list of coordinates with a maximum of 4 persons. It makes use of the build-in range-FFT, doppler-FFT and angle-FFT implementations, described in Section~\ref{sec:background}. Using these methods, a 2 dimentional heatmap can be generated. On the x-axis are all of the bins in the azimuth direction and on the y-axis are all of the bins in the range direction. Each bin contains information about the energy level in that bin. In other words, how much signal reflection has been observed in that bin location. 

% TODO add in schematic view of heatmap generation

\subsection{Noise removal}
Before persons can be detected, the amount of background noise needs to be removed. Because with less noise, the persons in front of the sensor will stand out more and will be easier to detect. Also, the signal strength will be improved. For the vital signs application the sensor is used for in this project, it can be assumed that the sensor will be in a static position, so there will be no background change. For each new background, a new calibration round needs to be done. During this calibration, the sensor will scan the space in front of it. It is very important that no other persons or objects other than static ones are at that moment in view of the sensor. The sensor will scan 64 and take the average of those scans. This noise map will both be saved on the device and returned to the computer via UART. In this way, the computer can send the values along with all of the other parameters when the sensor is restarted, and the calibration round doesn't need to be run again. When the calibration data is in place, it can be used to remove the noise from each new frame that is coming in.

% TODO add in picture with and without noise removal

\subsection{Peak detection}
This algorithm works by detecting peaks. During the testing phase, it was determined what the minimal peak height is for a person in the frame. The maximum heat in one heatmap bin is \emph{32000}. The minimal peak generated by a person in a frame is \emph{2000}. This is the first filter, only the peaks which are higher than the threshold are considered. 

For this application, it is assumed that each person is sitting approximately half a meter to one meter apart. This translates in the grouping functionality of the person detection algorithm. For each new peak that is found, it is checked if there has been another peak found within half a meter of the new peak. If that's the case, the peak with the biggest heat will be added to the list. This results in finding the biggest peak for each person, provided they sit half a meter apart. The biggest peak gives the information with the highest SNR value, such that the following algorithms will work in the most optimal form.

\begin{algorithm}
\caption{Person finding algorithm}\label{alg:person_finding_algorithm}
\begin{algorithmic}
\Require $heatmap$
\State $maxPeaks \gets list()$
\For{$x=0,1,\ldots,azimuthLength$}
    \For{$y=0,1,\ldots,rangeLength$}
        \State $bin \gets heatmap(x, y)$
        \If{$bin > 2000$}
            \If{$bin$ heat is larger than surrounding bins}
                \If{bin $c$ in $heatmap$ with distance $<$ 0.5 meter}
                    \If{$bin > c$}
                        \State Add $bin$ to $maxPeaks$
                        \State Remove $c$ from $maxPeaks$
                    \EndIf
                \Else
                    \State Add $bin$ to $maxPeaks$
                \EndIf
            \EndIf
        \EndIf
    \EndFor
\EndFor
\end{algorithmic}
\end{algorithm}

\subsection{Result}
In Algorithm~\ref{alg:person_finding_algorithm}, an overview of the algorithm in pseudo code can be found. The algorithm in itself is not very complicated, but it gets the job done for this use case. 

% TODO add pictures to display the result